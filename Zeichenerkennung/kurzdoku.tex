\documentclass[10pt,a4paper,ngerman,oneside]{scrartcl}

%%%%%%%%%%%%%%%%%%%%%%%%%%%%%%%%%%%%%%%%%%%%%%%%%%%%%%%%%%%%%
% Languages:

% Falls die Ausarbeitung in Deutsch erfolgt:
\usepackage[ngerman]{babel}
\usepackage[T1]{fontenc}
%\usepackage[latin1]{inputenc}
\usepackage[utf8]{inputenc}
\selectlanguage{ngerman}

\parskip 12pt plus 1pt minus 1pt
\parindent 0pt

%%%%%%%%%%%%%%%%%%%%%%%%%%%%%%%%%%%%%%%%%%%%%%%%%%%%%%%%%%%%%
% Meta informations:
\newcommand{\trauthor}{Dennis Fölster, Jim Martens}
\newcommand{\trtype}{} %{Proseminar} %{Seminar} %{Workshop}
\newcommand{\trcourse}{Praktikum Neuronale Netze}
\newcommand{\trtitle}{Zeichenerkennung mit Neuronalen Netzen}
\newcommand{\trmatrikelnummer}{} %
\newcommand{\tremail}{3foelste@inf, 2martens@inf}
\newcommand{\trinstitute}{Dept. Informatik -- Knowledge Technology, WTM}
\newcommand{\trwebsiteordate}{\url{http://www2.informatik.uni-hamburg.de/wtm/}}
%%%%%%%%%%%%%%%%%%%%%%%%%%%%%%%%%%%%%%%%%%%%%%%%%%%%%%%%%%%%%


\title{\trtitle}
\author{\trauthor}
\date{\today}


\begin{document}
	\maketitle
	\pagebreak

	\noindent
	Unser Quellcode ist aufgeteilt in 4 aufeinander aufbauende Module. MultiLayerNetwork und Presizer bilden die 
	Basis für den Recognizer, welcher die Basis für die GUI bildet.
	
	\section{MultiLayerNetwork}
		Bei der Klasse MultiLayerNetwork handelt es sich um ein einstellbares MLP. 
		Nur der Parameter \texttt{layout} muss zwingend angegeben werden. Hierbei Handelt es sich 
		um ein Tupel das jeweils die Größe der Layer beschreibt. Bei (4, 3, 1) wird
		dann beispielsweise ein Netzwerk mit 4 Eingabe- und 3 Hiddenneuronen sowie einem Ausgabeneuron 
		erzeugt. Der zweite Parameter \texttt{**options} nimmt alle weiteren Parameter, die übergeben
		werden, als ein Dictionary auf. 

		Die Parameter \texttt{transfer\_function} und \texttt{last\_transfer\_function} werden ausgewertet und erwarten eine
		Funktion, die dann jeweils als Aktivierungsfunktion zwischen den verschiedenen Layern verwendet wird.
		
		Innerhalb der Klasse MultiLayerNetwork wurden bereits einige Transferfunktionen wie z.B. die \texttt{sigmoid\_function} definiert.
		Sie sind als static definiert, damit sie der \texttt{\_\_init\_\_} Methode übergeben werden können. Alternativ hätten diese
		Funktionen jedoch auch außerhalb der Klasse definiert werden können. Allerdings wären sie dann beim Import der Klasse nicht
		vorhanden. Es können jedoch auch beliebige andere Funktionen als Parameter übergeben werden, solange sie als Eingabe ein Numpy-Array
		erwarten und auch wieder eines zurückgeben.

		Über die Methode \texttt{calc(input)} wird die Eingabe über alle Layer vorwärts propagiert und die Ausgabe des letzten Layers 
		zurückgegeben. Da unser MLP mit einem Bias arbeitet, wird für jede Layereingabe noch eine eins angehangen.
		Außerdem wird für jedem Layer Ein- und Ausgabe gespeichert, da diese für die eventuelle Rückwärtspropagation benötigt wird.
		Mit der Methode \texttt{train}, die eine Eingabe und das gewünschte Ergebnis erwartet, wird das Netz trainiert.
		Zuerst wird dafür \texttt{calc} aufgerufen und dann mittels dem gewünschtem Ergebnis und den gespeicherten Gewichten der
		Fehler zurückpropagiert. 
		Die Methode \texttt{train\_until\_fit} ermöglicht es ganze Datensätze zu trainieren. Sie erwartet ein Array aus 
		Paaren von Eingabe und gewünschter Ausgabe. Mit diesen Daten wird das Netz dann so lange trainiert, bis alle
		Eingaben das gewünschte Ergebnis liefern oder die Anzahl der maximalen Trainingsschritte erreicht ist. 

	\section{Presizer}
		In presizer.py befinden sich eine Reihe von Funktion zur Vorverarbeitung von Bilddateien.
		Die Funktion \texttt{getOptimizedImage(imagePath)} lädt die Bilddatei von der Festplatte und gibt ein auf den Inhalt zugeschnittenes 
		Bild in der Größe 20x30 Pixel zurück. Hierbei handelt es sich dann immer noch um ein PIL Image, welches über die Funktion
		\texttt{getDataFromImage(image)} in eine Liste aus Einsen und Nullen umgewandelt wird. Eine 0 gibt an, dass es sich bei dem Pixel um 
		Hindergrund handelt, wohingegen eine 1 bedeutet, dass es ein Teil des Symbols ist.

	\section{Recognizer} 
		Der Recogizer erstellt ein einfaches MLP mit einer versteckten Schicht, welches die Symbole erkennen soll.
		Über die Methode \texttt{train(self, folderpaths, learnrate, maxtrains)} wird dieses MLP trainiert. Bei \texttt{folderpaths} handelt es 
		sich um eine Liste mit Verzeichnissen, die Trainingsdaten enthalten. Alle .jpg Dateien in den Ordnern werden zum Training 
		verwendet. Der Anfangsbuchstabe der Datei muss dabei dem gewünschtem Ergebnis entsprechen.
		Mit \texttt{getResult(imgagePath)} bzw. \texttt{getResults(imgagePath)} werden die Ergebnisse für ein Bild ausgegeben. Die erste der beiden
		Funktionen gibt das Ergebnis als char zurück, während die zweite der Funktionen die drei wahrscheinlichsten Ergebnisse zusammen mit ihren
		Wahrscheinlichkeiten zurückgibt.

	\section{GUI}
		Die Klasse GUI erstellt mittels Pygame eine Oberfläche zur einfacheren Interaktion mit dem Recognizer.
		Auf der linken Hälfte wird immer, wenn die linke Maustaste gedrückt ist und sich die Maus bewegt, eine Linie 
		gezeichnet. Jedes Mal wenn der Mausbutton gelöst wird, wird der Recognizer mit dem momentanen Fensterinhalt als
		Eingabe aufgerufen und die Ausgabe auf der rechten Seite angezeigt. 
		Das Mausverhalten und Tastatureingaben werden regelmäßig in der Methode handleEvents abgefragt.


	
\end{document}